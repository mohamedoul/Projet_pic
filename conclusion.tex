\section* {Conclusion}
En somme, notre projet qui consistait en la réalisation d’un logiciel permettant  la gestion et la mise à jour des challenges au profit de la fédération Française de triathlon, nous a permis dans un premier temps de prendre contact avec le maître d’ouvrage afin de  déterminer et de statuer sur ses besoins fonctionnels. Il nous a permis, également,  de comprendre le processus de l’organisation des courses et de la mise à jour des challenges organisés par le Fédération Française de Triathlon. Par ailleurs, le projet était pour nous l’occasion de mettre en pratique les acquis théoriques relatifs aux méthodes de gestion de projet, essentiellement SCRUM, de réaliser la conception moyennant le langage de modélisation UML ainsi   que la manipulation de nouveaux langages informatiques essentiellement R et son serveur dédié Shiny.\\
En terme de perspectives, malgré le fait que langage R soit puissant en terme de calcul et de manipulation des statistiques de données volumineuses tout en diminuant la complexité des traitements,  néanmoins, il demeure limité en termes d’interfaces graphiques dynamiques, comparé à d’autres langages dédiés au web comme le PHP ou le JAVA EE. Étant donné que le nombre de courses organisées durant l’année est limité et que les challenges de chaque année sont traités séparément, il serait donc judicieux, dans le cadre de l’évolution du projet de le reprendre en utilisant un langage dédié au web en assurant la persistance donnée par une base de données et également d’implémenter le module relatif à l’organisation de la course afin que le système d’information soit entièrement informatisé. 
