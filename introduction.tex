	\addcontentsline{toc}{section}{Introduction}
	\section*{Introduction}
	Dans un contexte marqué par l’émergence du web, le besoin accru d'informatisation et d'automatisation des gestions, nous avons développé un site web pour permettre à la ligue des Hauts-de-France de mettre à jour les challenges organisés durant chaque année et aux participants de consulter leurs résultats.
	En effet, le long du processus de développement allant des premières phases de la définition des besoins fonctionnels du client jusqu’aux phases de création et de test, nous avons fixé trois lignes de conduite, à savoir :
	\begin{enumerate}
	\item Le site web doit satisfaire les besoins de ses utilisateurs et des administrateurs
	\item Il doit répondre aux qualités requises en matière de maintenabilité et de portabilité
	\end{enumerate}
	Afin de mettre la lumière sur les différentes démarches suivies pour l’aboutissement de ce travail, nous avons scindé ce rapport en trois grandes parties. Dans cette optique, la première sera réservée à l’illustration du contexte général du projet, tandis que la deuxième sera consacrée à l’étude analytique des différents processus de données et des phases de notre projet. Alors que la 3ème partie sera dédiée à la présentation des langages utilisés pour la réalisation du site web, de ses interfaces ainsi que des tests mis en œuvre pour s’assurer de son bon fonctionnement. Par ailleurs, il convient de préciser que ce rapport sera joint d’un guide d’utilisation qui sera mis à la disposition aux administrateurs de la ligue.