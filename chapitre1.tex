\section {Contexte général du projet}
	
	Cette partie traitera du contexte général de notre Projet d’Innovation et de Conception, en mettant l’accent essentiellement sur une brève présentation du maître d'ouvrage et de ses exigences fonctionnelles. Également, la méthodologie de travail adoptée ainsi que la planification du projet seront mises en avant.  
	
	\subsection{La ligue des Hauts-de-France de triathlon }
	
	La Ligue des Hauts-de-France de triathlon est un organe décentralisé de la F.F.TRI. Elle est administrée par un Comité Directeur et un Bureau Directeur composés de membres élus en son sein. 
	Pour fonctionner, la ligue des Hauts-de-France de triathlon , a constitué plusieurs commissions dont chacune est présidée par un membre du Comité Directeur et composée de membres de la ligue volontaires et choisis pour leurs compétences.
	Celle-ci n'a été crée que récemment, car elle fait suite à la fusion des régions Picardie et Nord-Pas-de-Calais.
	L'assemblée générale s'est alors tenue le 03 février 2018, \href{http://triathlonhdf.fr/samedi-03-fevrier-2018-la-ligue-des-hauts-de-france-de-triathlon-est-creee/}{un court article a été rédigé sur leur nouveau site web à l'adresse triathlonhdf.fr} \cite{ref1}
	
	\subsection{Description du processus de l’organisation des challenges }
	L’organisateur de la course envoie les informations de tous les participants ainsi que leurs classements à la ligue. Ces fichiers sont envoyés sous formats PDF, Word, Tableurs Excel voire même papier.  Ces données sont envoyées à la ligue sans aucun traitement ou filtrage. Par conséquent, les administrateurs de la ligue ont pour première tâche de supprimer les participants qui ne sont pas licenciés au niveau de la ligue, et également de traiter le problème des homonymes d’une manière manuelle. Tâche qui demeure difficile et demande du temps.
	Une fois que les données sont actualisées, les fichiers challenge sont mis à jour et les classements globaux sont établis. L’ensemble de ces opérations sont faites moyennant l'utilisation des fichiers Excel.
	
	
	\subsection{Les modalités du challenge  }
	Les fichiers challenges organisés sont soumis à plusieurs règles qui sont répertoriées comme suit :
	\begin{itemize} 
	\item 	Toutes les épreuves individuelles de cross duathlon, duathlon, aquathlon, cross triathlon et triathlon organisées par la Ligue NPdC, distance XS, S, M, L ou XL sont retenues pour le challenge.
	\item 	Les organisations s'inscrivant dans le circuit des épreuves du challenge doivent se soumettre au cahier des charges proposés par la Ligue.
	 \item  Seront retenues au challenge de l’année suivante uniquement les épreuves qui rempliront les contraintes et fourniront les résultats au format prévu  au plus tard dans la semaine suivant l’épreuve.
	
	\end{itemize} 
\newpage
	 Attribution des points. Un classement individuel par catégorie et par sexe est réalisé à l’issue de chacune des épreuves. Seule la distance la plus longue accessible à la catégorie sera prise en compte dans l’obtention des points.\\
	L’attribution des points se fait en fonction du classement par catégorie et sexe. 
	
	Les tableaux figurant ci-dessous récapitulent le nombre de points attribués selon le classement du participant :
	\begin{enumerate}
	   \item Pour les adultes
	\begin{figure}[h!]
	   \center
	   \includegraphics[scale=0.9]{img/points_categorie_adultes.png}
	   \caption {Les points attribués aux participants de la catégorie adultes selon leurs classements}
	\end{figure}
	
	Au-delà de la 142ème place, 1 point est attribué à chacun des participants. Les points obtenus sont pondérés par un coefficient dépendant de la distance de l’épreuve : 
	
	\begin{itemize} 
	 	\item 0,5 pour un XS ; 
	 	\item 0,75 pour un S ;
	 	\item 1 pour un M ;  
	 	\item 1,5 pour un L ;
	 	\item 2 pour un XL ou XXL.
	\end{itemize} 
	   \item Pour les jeunes
	\begin{figure}[!h]
	   \center
	   \includegraphics[scale=1]{img/points_categorie_jeunes.png}
	   \caption {Les points attribués aux participants de la catégorie jeunes selon leurs classements}
	\end{figure}
	
	\end{enumerate}

	\newpage
	Au-delà de la 20ème place, un point est attribué à chacun des participants. Les abandons et les absents ne marquent pas de point. 
	L’ensemble des règles susmentionnées sont récapitulées :
	\begin{figure}[!h]
	   \center
	   \includegraphics[scale=0.8]{img/recapitulatif_modalite_challenge.png}
	   \caption {Récapitulatif des modalités du challenge}
	\end{figure}
	
	\newpage
	\subsection{Expression des besoins fonctionnels}
	L’outil développé dans le cadre de ce projet doit répondre aux fonctionnalités définies en concertation avec le client lors des réunions. Ainsi l’ensemble de ces fonctionnalités sont répertoriées comme suit :
	\begin{itemize} 
	\item Supprimer les participants d'une course qui ne sont pas licenciés à la ligue	
	\item Pallier le problème des homonymes 
	\item Création d’un challenge
	\item La mise à jour d’un challenge
	\item Affichage des résultats globaux des courses
	\item Afficher les statistiques individuelles avec les graphes correspondants
	\item Effectuer des recherches des participants selon plusieurs critères
	\item Toutes ces opérations doivent être réalisées via une interface graphique conviviale.
	\end{itemize} 
	
	\subsection {Méthodologie de travail }
	Afin de garantir la réussite du projet, il a été décidé de recourir à la méthode agile SCRUM. En effet, celle-ci consiste à découper le projet en boîtes de temps appelées sprints. Chaque sprint commence par une estimation suivie d'une planification opérationnelle. Le sprint se termine par une démonstration de ce qui a été achevé. Avant de démarrer un nouveau sprint, l'équipe réalise une rétrospective. Cette technique analyse le déroulement du sprint achevé, afin d'améliorer ses pratiques. Le flot de travail de l'équipe de développement est facilité par son auto-organisation. 
	Pour mettre cette méthode en pratique. Une réunion est organisée chaque jeudi après-midi avec nos encadrants pour discuter des avancements de la semaine en cours et en de déterminer les tâches à effectuer durant la semaine à venir. Par ailleurs, l’ensemble des tâches à réaliser sont réparties sur les membres de l’équipe afin de converger les efforts pour la réussite du projet. 
	
	\subsection{Diagramme de Gantt}
	La planification fait partie des phases d’avant-projet. Elle consiste à prévoir le déroulement du projet tout au long de ses phases. Grâce aux réunions tenues avec nos encadrants et le client, nous avons été éclairés sur les différentes étapes du projet.
	Ainsi le diagramme ci-après résume les différentes étapes ainsi que le temps qui leur a été alloué.
	\newpage
\begin{sidewaysfigure}
	   \center
	   \includegraphics[scale=0.8]{img/diagramme_de_gantt.png}
	   \caption {Le diagramme de Gantt du projet}	
\end{sidewaysfigure}
	\newpage
   Comme il est clairement illustré dans ce diagramme en figure 4, le déroulement du projet s'est fait en plusieurs étapes à savoir:
 \begin{itemize}
\item La définition des besoins du maître d'ouvrage
\item Le choix du langage R et du Serveur Shiny pour l'implémentation du projet
\item La formation sur les langages à utiliser qui a duré tout le long du processus du développement du projet
\item Implémentation des fonctionnalités souhaitées par le client en se basant sur le code R fourni
\item Effectuer des tests nécessaires sur chaque fonctionnalité
\item Déploiement de l'application sur serveur de test et réaliser le test d'acceptation sur l'ensemble des fonctionnalités
\item La documentation du code: cette étape est cruciale pour faciliter la compréhension du code, l'utilité de chaque fonction et également pour faciliter les étapes futures  de l'évolution du code.
\item La rédaction du rapport qui est une étape faite en parallèle des autres.
 \end{itemize}